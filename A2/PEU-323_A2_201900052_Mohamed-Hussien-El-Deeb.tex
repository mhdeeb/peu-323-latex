\documentclass[12pt]{article}
\usepackage[svgnames,x11names,table]{xcolor}
\usepackage{hyperref}
\usepackage{graphicx}
\usepackage{parskip}
\usepackage{float}
\usepackage{amsmath}
\usepackage{amssymb}
\usepackage{enumitem}
\usepackage[thicklines]{cancel}

\hypersetup{
    colorlinks,
    citecolor=black,
    filecolor=black,
    linkcolor=RoyalBlue4,
    urlcolor=RoyalBlue4,
}

\title{PEU 323 Assignment 2}
\author{Mohamed Hussien El-Deeb (201900052)}
\date{\today}

\begin{document}

\maketitle
\tableofcontents

\section{Problem 1}

\subsection{Problem}

Consider a particle which is free to be anywhere, with equal probability, on a line segment of length \(L\).

\renewcommand{\labelenumi}{(\alph{enumi})}
\begin{enumerate}
    \item Find the normalized probability distribution for such a particle.
    \item Calculate \(\langle x\rangle\) and \(\sigma_{x}\).
    \item Calculate the probability of finding the particle within \(\sigma_{x}\) of \(\langle x\rangle\).
    \item What are the dimensions of the probability density?
\end{enumerate}

\subsection{Solution}

\section{Problem 2}

\subsection{Problem}

Buffon's Needle: A needle of length \(l\) is dropped at random on a sheet of paper with parallel lines a distance \(l\) apart. What is the probability that it crosses a line?

\subsection{Solution}

\section{Problem 3}

\subsection{Problem}

Find the probability distribution for the momentum of a harmonic oscillator with angular frequency \(\omega\).

\subsection{Solution}

\section{Problem 4}

\subsection{Problem}

Consider the probability density for the location of the electron inside the Hydrogen atom:

\begin{equation}
    \rho(r)=A e^{-2 r / a_{0}},
\end{equation}

where \(a_{0}\) is the Bohr radius.

\renewcommand{\labelenumi}{(\alph{enumi})}
\begin{enumerate}
    \item Find \(A\) which normalizes this probability distribution.
    \item
          \textit{Hint:}

          \begin{equation}
              \int_{0}^{\infty} e^{-x} x^{n} d x=n!.
          \end{equation}
    \item Calculate the probability for the electron to be found in a sphere, centered about the origin, of radius \(b_{0}\), with \(b_{0}<<a_{0}\).

          \textit{P.S.:} You can do this calculation exactly or approximately. The approximate one is much easier.
\end{enumerate}

\subsection{Solution}

\section{Problem 5}

\subsection{Problem}

Consider the map

\begin{equation}
    f: \mathbb{C} \rightarrow \mathbb{R}_{2 \times 2}
\end{equation}

\[
    x + iy \mapsto
    \begin{pmatrix}
        x & -y \\
        y & x
    \end{pmatrix}
\]

from the complex numbers to the set of \(2 \times 2\) real matrices.

\renewcommand{\labelenumi}{(\alph{enumi})}
\begin{enumerate}
    \item Show that this map is an isomorphism. That is, show that it is invertible and that for \(z_{1}, z_{2} \in \mathbb{C}\),

          \begin{equation}
              f\left(z_{1} z_{2}\right)=f\left(z_{1}\right) f\left(z_{2}\right)
          \end{equation}

          and hence prove that, in two dimensions, rotations commute.
    \item Prove De Moivre's formula for complex numbers:
          \begin{equation}
              (r(\cos (\theta)+i \sin (\theta)))^{n}=r^{n}(\cos (n \theta)+i \sin (n \theta))
          \end{equation}

          and hence prove that

          \begin{equation}
              \left(\begin{array}{cc}
                  \cos (\theta) & -\sin (\theta) \\
                  \sin (\theta) & \cos (\theta)
              \end{array}\right)^{n}=\left(\begin{array}{cc}
                      \cos (n \theta) & -\sin (n \theta) \\
                      \sin (n \theta) & \cos (n \theta)
                  \end{array}\right)
          \end{equation}
\end{enumerate}

\subsection{Solution}

\bibliographystyle{plain}
\bibliography{references}
\nocite{El-Deeb_PEU-323_Assignments}

\end{document}